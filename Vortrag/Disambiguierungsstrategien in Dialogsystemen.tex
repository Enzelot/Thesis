
\documentclass{beamer}
\usepackage[utf8]{inputenc}
\usepackage[ngerman]{babel}
\usepackage{multicol}
\usepackage[utf8]{inputenc}
\usepackage[T1]{fontenc}
\usepackage[ngerman]{babel}
\usepackage{amsmath}
\usepackage{amsfonts}
\usepackage{amssymb}
\usepackage{paralist}
\usepackage{color}
\usepackage{xcolor}
\usepackage{lmodern}
\usepackage{graphicx}
\usepackage[mediumspace,mediumqspace,squaren]{SIunits} %SI-Einheiten
\usepackage[version=3]{mhchem} %chemische Formeln
\usepackage{tabularx, calc} %Tabellen
\usepackage{multirow} %Tabelle: mehrere Spalten vereinigen
\usepackage{booktabs} %Verbesserung der Tabellenqualität
\usepackage{caption} %Abbildungs- und Tabellenbeschriftung
\usepackage{array} %Matheumgebung ähnlich tabular
\usepackage{pgf}
\usepackage{dcolumn}
\usepackage{beamerthemeshadow}
\beamersetuncovermixins{\opaqueness<1>{25}}{\opaqueness<2->{15}}
\begin{document}
\title{Bachelorarbeit}
\subtitle{Disambiguierungsstrategien in Dialogsystemen }  
%\title{Disambiguierungsstrategien in Dialogsystemen }
%\subtitle{Bachelorarbeit}
\author{Lena Enzweiler}
\institute{Universität des Saarlandes}
\date{\today} 

\beamertemplatenavigationsymbolsempty

\setbeamertemplate{footline}[frame number]

\frame{\titlepage} 

 \section{Einleitung}
\subsection{Vorahnung}
\frame{\frametitle{Vorahnung}
\textsl{"The wise man avoids evil by anticipating it" (Publilius Syrus)}
\newline

Vorahnung ist lebenswichtig
\begin{itemize}
\item kein Halten von gef\"ahrlichen Tieren als Haustiere
\item keine Spazierg\"ange bei Gewitter                                                                                                                                                                                                                                                                                                                                                                                                                                                                                                                                                                                                                                                                                                                                                                                                                                                                                                                                                                                                                                                                                                                                                                                                                                                                               
\item Reflexe aus\"uben
\end{itemize} 
}

\frame{\frametitle{Vorahnung in der Sprache}

Gibt es Vorahnungen auch in der Sprache?
\begin{block}{1. Beispiel}
Der Einbrecher konnte den Familiensafe direkt finden.\\
Dieser befand sich nat\"urlich hinter einem ...  
\end{block}

Erwartet man hier ein bestimmtes Wort?\\
$\Rightarrow$ 83\% der Befragten erahnten \textbf{Bild/Gem\"alde}

\begin{block}{2. Beispiel}
Der Einbrecher konnte den Familiensafe direkt finden.\\
Dieser befand sich nat\"urlich hinter einer großen ...  
\end{block}

Die Befragten sind hier überrascht über das Auftreten der Wörter \textbf{einer} und \textbf{großen}
da diese nicht im Genus mit dem erwarteten Wort \textbf{Bild} übereinstimmt\\
$\Rightarrow$ Reaktionen am EEG und an der Lesezeit erkennbar
}

\subsection{Vorahnung in der Sprache}
\frame{\frametitle{Vorahnung in der Sprache}

Vorahnung spielt auch in der Sprache eine wichtige Rolle
\newline
\begin{itemize}
\item Einsatz in Konversationen
\item Satz des Gespr\"achspartners beenden
\item Priming
\item Holzwegeffekt
\end{itemize} 
}

\frame{\frametitle{Inhaltsverzeichnis}\tableofcontents}

\subsection{Fokus der Studie}
\frame{\frametitle{Fokus der Studie}
Nutzen Leser und Hörer zum Vorahnen von auftretenden W\"ortern Informationen aus dem Diskurs?
\newline
\begin{block}{1. Beispiel}
Der Einbrecher konnte den Familiensafe direkt finden.\\
Dieser befand sich nat\"urlich hinter einem ...  
\end{block}

Kann man hier ein bestimmtes Wort erwarten?\\
$\Rightarrow$ 83\% der Befragten erahnten \textbf{Bild/Gem\"alde}
}

\frame{\frametitle{Fokus der Studie}
\begin{itemize}
\item Leser und H\"orer m\"ussen zur Vorahnung bestimmter W\"orter entsprechende Situationen erkennen. \\
\item Dabei wird das Wissen  \"uber die Welt, den Sprecher, die Kommunikation und der Sprache vorausgesetzt
\end{itemize}
\begin{block}{2. Beispiel}
Der Ehemann war so w\"utend, dass er ihre Lieblingsvase gegen die Wand warf. \\
Sp\"ater tat es ihm leid, dass die Vase ... 
\end{block}
Erkennt man hier die Situation wei{\ss} man, dass die Vase zerbrochen sein muss und erwartet ein entsprechendes Wort
}

\frame{\frametitle{Fokus der Studie}

Wie wirken sich solche durch Vorahnungen verursachten Reaktionen im EEG und auf Lesezeiten aus?
\newline
\newline
Zur Analyse wurden drei verschiedene Experimente durchgeführt:

\begin{itemize}
\item Experiment 1: Hören mit Diskurs
\item Experiment 2: Hören ohne Diskurs
\item Experiment 3: Lesen mit Diskurs
\end{itemize}

}



\section{Experimente} 
\subsection{Experiment 1}
\frame{\frametitle{Experiment 1}
\begin{block}{Forschungsfrage}
Nutzen Hörer Informationen aus dem Kontext um bestimmte Wörter vorauszuahnen?
 \end{block}
\begin{itemize}

\item getestet an Hand von Ministorys\\
\textit{Der Einbrecher konnte den Familiensafe direkt finden.}\\
\textit{Dieser befand sich nat\"urlich hinter einem großen Bild.}
\item diese wurden mit bestimmten Adjektiven modifiziert
		\begin{itemize}
		\item Adjektive die im Genus mit dem zu erwartenden Nomen übereinstimmen
		\item Adjektive die nicht im Genus mit dem zu erwartenden Nomen übereinstimmen
		\end{itemize}
\item Reaktionen der Hörer am EEG gemessen
\end{itemize} 
} 

\frame{\frametitle{Was wurde erwartet}
\begin{itemize}
\item Übereinstimmung mit der Vorahnung
$\Rightarrow$ Kein Effekt erwartet
\item keine Übereinstimmung mit der Vorahnung
$\Rightarrow$ Effekt auf dem Adjektivsuffix und dem Nomen erwartet
\end{itemize}
}




\frame{\frametitle{Versuchspersonen}
\begin{itemize}
\item 24 Rechtshänder mit niederländischer Muttersprache
\item 18 Frauen, 6 Männer
\item vom Max Planck Institute für Psycholingusitik in Nijmege
\item keine neurologischen Beeinträchtigungen oder Einnahme von Neuroleptika
\end{itemize}
}

\frame{\frametitle{Adjektive im Niederländischen}
Jedes niederländische Nomen bestimmt in einem unbestimmten Satz das Suffix des Adjektivs\\
\begin{itemize}
\item en \textbf{groot}$_{neut}$ schilderij = ein großes Bild ( neutrales Null Suffix )
\item en \textbf{grote}$_{gel}$ boekenkast = ein großes Bücherregal ( geläufiges -e Suffix)
\end{itemize}
}

\frame{\frametitle{Materialien}
\begin{itemize}
\item 74 kritische Ministorys (mit hoher Vorahnung) mit 56 normalen Ministories (mit geringer Vorahnung)  gemischt
\item bestehend aus einem Kontext- und einem kritischen Zielsatz 
\item Der Einbrecher konnte den Familiensafe direkt finden. (Kontext)\\
\textbf{Dieser befand sich nat\"urlich hinter einem großen Bild} (Ziel)
\item Ministories wurden mit einem normalen Tempo und einer normalen Intonation aufgezeichnet und abgespielt
\item Auf jedes Adjektiv folgte ein grammatikalisch korrektes Nomen
	\begin{itemize}
	\item en groot$_{neut}$ \textbf{schilderij}
	\item en grote$_{gel}$ \textbf{boekenkast}
	\end{itemize}
\end{itemize}
}


\frame{\frametitle{Ablauf}
\begin{itemize}
\item Versuchspersonen saßen in schalldämpfender Kabine und bekamen Stimuli über Kopfhörer
\item Kurzes Training
\item 5 Blöcke von je 15 Minuten präsentiert
\item Präsentation: 
	\begin{itemize}
	\item 300ms Ankündigung
	\item 700ms Pause
	\item Kontextsatz
	\item 1000ms Pause
	\item Zielsatz
	\end{itemize}
\item 500ms vor dem Zielsatz wurde die Aufnahme des EEGs angekündigt (Bewegungen vermeiden)
\end{itemize}
}


\frame{\frametitle{Ergebnis}
\begin{itemize}
\item 3 kritische Reize betrachtet:
	\begin{itemize}
	\item Adjektiv
	\item Adjektivflexion
	\item Nomen
	\end{itemize}
\item durchschnittlicher Kurvenverlauf pro Versuchsperson für den jeweiligen Reiz berechnet
\item anschließend durchschnittlicher Kurvenverlauf aller Versuchspersonen berechnet
\end{itemize}
}


\frame{\frametitle{Ergebnis: Adjektiv}
\begin{itemize}
\item Durchschnitt der ereigniskorrelierten Potentiale für das akustische Einsetzen des kritischen Adjektiv
\item übereinstimmend mit der Vorahnung des Nomens $\rightarrow$ durchgezogene Linie
\item nicht übereinstimmend mit der Vorahnung des Nomens $\rightarrow$ punktierte Linie
\item Beispielzielsatz: \textit{Dieser befand sich nat\"urlich hinter einem großen Bild.}

%\begin{figure}[h]
%	\centering
%        \fbox{\includegraphics[scale=0.25]{adjectiv}}
%\end{figure}
\end{itemize}
}

\frame{\frametitle{Ergebnis: Adjektiv}


\begin{itemize}
\item kein klarer Effekt erkennbar, der durch das Adjektiv hervorgerufen wird
\end{itemize}
}


\frame{\frametitle{Ergebnis: Adjektivflexion}
\begin{itemize}
\item Durchschnitt der ereigniskorrelierten Potentiale für das akustische Einsetzen der kritischen Adjektivflexion
\item übereinstimmend mit der Vorahnung des Nomens $\rightarrow$ durchgezogene Linie
\item nicht übereinstimmend mit der Vorahnung des Nomens $\rightarrow$ punktierte Linie
\item Beispielzielsatz: \textit{Dieser befand sich nat\"urlich hinter einem großen Bild.}

\end{itemize}
}

\frame{\frametitle{Ergebnis: Adjektivflexion}


\begin{itemize}
\item Positivierung bei ca. 50ms von Flexion ausgelöst, die nicht mit der Vorahnung übereinstimmt
\end{itemize}

}

\frame{\frametitle{Ergebnis: Nomen}
\begin{itemize}
\item Durchschnitt der ereigniskorrelierten Potentiale für das akustische Einsetzen des Nomens
\item vorgeahntes Nomen $\rightarrow$ durchgezogene Linie
\item nicht vorgeahntes Nomen (Alternativnomen)  $\rightarrow$ punktierte Linie
\item Beispielzielsatz: \textit{Dieser befand sich nat\"urlich hinter einem großen Bild.}

\end{itemize}
}

\frame{\frametitle{Ergebnis: Nomen}


\begin{itemize}
\item nicht vorgeahnte Alternativnomen lösen großen N400 Effekt bei ca. 350-400ms aus
\end{itemize}

}


\frame{\frametitle{Zusammenfassung Experiment 1}
\begin{itemize}
\item Effekt durch Adjektivflexion ausgelöst bevor das Nomen präsentiert wurde\\
\item N400 Effekt bei unerwarteten Nomen ausgelöst
\item$\Rightarrow$ Diskurs Informationen lassen Hörer tatsächlich bestimmtes Wort im entfaltenden Satz erwarten
\item Versuchspersonen haben keine Manipulation der Daten bemerkt
$\Rightarrow$ Vorahnungen, sowie das Widerrufen dieser nach Auftreten unpassender Adjektive geschieht unbewusst und routinemäßig
\end{itemize}       
}

%%%%%%%%%%%%%%%%%%%%%%%%%%%%%%%%%%%%%%%%%%%
%%%%%%%%%%%%%%%%%%%%%%%%%%%%%%%%%%%%%%%%%%%
%%%%%%%Experiment 1 fertig%%%%%%%%%%%%%%%%%
%%%%%%% To go: Experiment 2%%%%%%%%%%%%%%%%
%%%%%%%%%%%%%%%%%%%%%%%%%%%%%%%%%%%%%%%%%%%
%%%%%%%%%%%%%%%%%%%%%%%%%%%%%%%%%%%%%%%%%%%

\subsection{Experiment 2}
\frame{\frametitle{Experiment 2}
\begin{block}{Forschungsfrage}
Erwarten Hörer bestimmte Wörter in einem Satz ohne weiteren Kontext?
 \end{block}
\begin{itemize}
\item getestet mit den Zielsätzen aus Experiment 1 \\
$\Rightarrow$ Diskurs fehlt

\item Reaktionen der Hörer am EEG gemessen
\end{itemize} 
} 

%\frame{\frametitle{Was erwartet wurde}
%\begin{itemize}
%\item Stimmt das vorgeahnte Nomen mit dem Adjektiv im Genus überein
%$\Rightarrow$ Kein besonderer Effekt erwartet
%\item Stimmt das vorgeahnte Nomen nicht mit dem Adjektiv im Genus überein
%$\Rightarrow$ N400 Effekt in Höhe des Adjektivs erwartet
%\end{itemize}
%}




\frame{\frametitle{Versuchspersonen}
\begin{itemize}
\item 24 Rechtshänder mit niederländischer Muttersprache
\item 18 Frauen, 6 Männer
\item vom Max Planck Institute für Psycholingusitik in Nijmege
\item keine neurologischen Beeinträchtigungen oder Einnahme von Neuroleptika
\item keine Person hat an Experiment 1 teilgenommen
\end{itemize}
}


%\frame{\frametitle{Materialien}
%\begin{itemize}
%\item Insgesamt 120 Zielsätze aus Experiment 1\\
%$\Rightarrow$ Diskurs fehlt
%\end{itemize}
%}


\frame{\frametitle{Ablauf}
\begin{itemize}
\item Versuchspersonen saßen in schalldämpfender Kabine und bekamen Stimuli über Kopfhörer
\item Kurzes Training
\item 5 Blöcke von je 15 Minuten präsentiert
\item Präsentation: 
	\begin{itemize}
	\item 300ms Ankündigung
	\item 1200ms Pause
	\item Zielsatz
	\end{itemize}
\item 1000ms vor dem Zielsatz wurde die Aufnahme des EEGs angekündigt (Bewegungen vermeiden)
\end{itemize}
}


\frame{\frametitle{Ergebnis}
\begin{itemize}
\item 2 kritische Reize betrachtet:
	\begin{itemize}
	\item Adjektivflexion
	\item Nomen
	\end{itemize}
\item durchschnittlicher Kurvenverlauf pro Versuchsperson für den jeweiligen Reiz berechnet
\item anschließend durchschnittlicher Kurvenverlauf aller Versuchspersonen berechnet
\end{itemize}
}


\frame{\frametitle{Ergebnis: Adjektivflexion}
\begin{itemize}
\item Durchschnitt der ereigniskorrelierten Potentiale für das akustische Einsetzen der kritischen Adjektivflexion
\item übereinstimmend mit der vorherigen Vorahnung des Nomens $\rightarrow$ durchgezogene Linie
\item nicht übereinstimmend mit der vorherigen Vorahnung des Nomens $\rightarrow$ punktierte Linie
\item Beispielzielsatz: \textit{Dieser befand sich nat\"urlich hinter einem großen Bild.}

\end{itemize}
}

\frame{\frametitle{Ergebnis: Adjektivflexion}

%\begin{figure}[htbp]
%	\begin{center}
%		\begin{minipage}[t]{0.4\linewidth}
%			\centering
%			\includegraphics[width=\linewidth]{adjectivflexion}
%			\caption{mit Kontext}
%		\end{minipage}
%		\qquad
%%		\begin{minipage}[t]{0.4\linewidth}
	%		\centering
	%		\includegraphics[width=\linewidth]{adjektivflexion2}
	%		\caption{ohne Kontext}
	%	\end{minipage}
	%\end{center}
%\end{figure}

\begin{itemize}
\item mit Kontext: positiver Effekt bei der Adjektivflexion, die nicht mir dem vorherigen vorgeahnten Wort übereinstimmt
\item ohne Kontext: Effekt bleibt aus
\end{itemize}

}

\frame{\frametitle{Ergebnis: Nomen}
\begin{itemize}
\item Durchschnitt der ereigniskorrelierten Potentiale für das akustische Einsetzen des Nomens
\item vorheriges vorgeahntes Nomen $\rightarrow$ durchgezogene Linie
\item nicht vorheriges vorgeahntes Nomen (Alternativnomen)  $\rightarrow$ punktierte Linie
\item Beispielzielsatz: \textit{Dieser befand sich nat\"urlich hinter einem großen Bild.}

\end{itemize}
}

\frame{\frametitle{Ergebnis: Nomen}



\begin{itemize}
\item mit Kontext: N400 Effekt bei nicht vorgeahnten Nomen
\item ohne Kontext: N400 Effekt bleibt aus (Abschweifung zu spät)
\end{itemize}


}


\frame{\frametitle{Zusammenfassung Experiment 2}
\begin{itemize}
\item (vorheriges) überraschendes Adjketivsuffix:
	\begin{itemize}
	\item mit Kontext: Positivierung bei ca. 50 ms von Flexion ausgelöst
	\item ohne Kontext: kein Effekt\\
	\item $\Rightarrow$ Positivierung vom Kontext abhängig
	\end{itemize}
	
\item (vorheriges) nicht vorgeahntes Nomen
\begin{itemize}
\item mit Kontext: N400 Effekt
\item ohne Kontext: kein N400 Effekt
\item $\Rightarrow$ N400 Effekt vom Kontext abhängig


\end{itemize}
\end{itemize}

}


%%%%%%%%%%%%%%%%%%%%%%%%%%%%%%%%%%%%%%%%%%%
%%%%%%%%%%%%%%%%%%%%%%%%%%%%%%%%%%%%%%%%%%%
%%%%%%% Experiment 1 fertig %%%%%%%%%%%%%%%
%%%%%%% To go: Experiment 3 %%%%%%%%%%%%%%%
%%%%%%%%%%%%%%%%%%%%%%%%%%%%%%%%%%%%%%%%%%%
%%%%%%%%%%%%%%%%%%%%%%%%%%%%%%%%%%%%%%%%%%%

\subsection{Experiment 3}
\frame{\frametitle{Experiment 3}
\begin{block}{Forschungsfrage}
Nutzen Leser Informationen aus dem Kontext um bestimmte Wörter vorauszuahnen?
 \end{block}
\begin{itemize}
\item getestet an Hand von Ministorys
\item diese wurden mit bestimmten Adjektiven modifiziert
		\begin{itemize}
		\item Adjektive die im Genus mit dem zu erwartenden Nomen übereinstimmen
		\item Adjektive die nicht im Genus mit dem zu erwartenden Nomen übereinstimmen
		\end{itemize}
\item Reaktionen der Leser an Lesezeit bewertet
\end{itemize} 
} 

\frame{\frametitle{Was wurde erwartet}
\begin{itemize}
\item Übereinstimmung mit der Vorahnung
$\Rightarrow$ Normale Lesezeit
\item keine Übereinstimmung mit der Vorahnung
$\Rightarrow$ Verzögerte Lesezeit
\end{itemize}
}





\frame{\frametitle{Versuchspersonen}
\begin{itemize}
\item 24 Rechtshänder mit niederländischer Muttersprache
\item 21 Frauen, 3 Männer
\item von der Universität Amsterdam
\item keine hat am Experiment 1 oder Experiment 2 teilgenommen
\end{itemize}
}


\frame{\frametitle{Materialien}
\begin{itemize}
\item 40 Ministories mit 56 normalen Ministories gemischt
\item drei Wörter zwischen kritischen Adjektiv und Nomen
\item Der Einbrecher konnte den Familiensafe direkt finden. (Kontext)\\
Dieser befand sich nat\"urlich hinter einem großen aber eher unauffälligen Bild (Ziel)
\end{itemize}
}


\frame{\frametitle{Ablauf}
\begin{itemize}
\item Präsentation Wort für Wort
\item Nächstes Wort durch Klicken erschienen
\item Position des Wortes im Satz angezeigt
\item 4 Blöcke, jeweils durch Pausen getrennt
\end{itemize}
}


\frame{\frametitle{Analyse der Lesezeit}
\begin{itemize}
\item Insgesamt neun Wortpositionen betrachtet
 \begin{table}
	\tiny
		\begin{tabular}{p{0,6cm}>{\centering\arraybackslash}p{0.6cm}>{\centering\arraybackslash}p{0.6cm}>{\centering\arraybackslash}p{0.6cm}>{\centering\arraybackslash}p{0.5cm}>{\centering\arraybackslash}p{0.9cm}>{\centering\arraybackslash}p{0.5cm}>{\centering\arraybackslash}p{0.6cm}>{\centering\arraybackslash}p{0.6cm}>{\centering\arraybackslash}p{0.9cm}}
	 Position & cw-4 & cw-3  & cw-2 &  cw-1 & Adjektiv & cw+1 & cw+2 & cw+3 & Nomen\\
	\midrule
	 Beispiel & ...was & situated  & behind &  a & big-Suffix & but & rather & unobtru-sive & painting/ bookcase\\
	\end{tabular}
	\end{table}
\item Für jede Position wurde pro Person und Satz die durchschnittliche Lesezeit berechnet
\end{itemize}
}


\frame{\frametitle{Ergebnis: Lesezeiten}
\begin{table}
	\tiny
		\begin{tabular}{p{1cm}>{\centering\arraybackslash}p{0.6cm}>{\centering\arraybackslash}p{0.6cm}>{\centering\arraybackslash}p{0.6cm}>{\centering\arraybackslash}p{0.5cm}>{\centering\arraybackslash}p{0.9cm}>{\centering\arraybackslash}p{0.5cm}>{\centering\arraybackslash}p{0.6cm}>{\centering\arraybackslash}p{0.6cm}>{\centering\arraybackslash}p{0.9cm}}
	\toprule
	 Positionen & cw-4 & cw-3  & cw-2 &  cw-1 & Adjektiv & cw+1 & cw+2 & cw+3 & Nomen\\
	\midrule
	 Beispiel & ...was & situated  & behind &  a & big-Suffix & but & rather & unobtru-sive & painting/ bookcase\\
	konsistent & 408 & 365  & 360 & 327 & 342 & 350 & 373 & 407 & 498\\
	inkonsistent  & 402 & 364 & 363 & 336 & 351 & 352 & 374 & 425 & 598\\
	Effektgröße & -6 &  -1 & 3 & 9 & 9 & 2 & 1 & 18 & 100\\
	\bottomrule
	\end{tabular}
	\end{table}
\begin{itemize}
\item kein Effekt vor dem kritischen Adjektiv zu erkennen
\item großer Effekt auf dem Nomen zu erkennen
\item Unerwartet: Kein Effekt auf dem kritischen Adjektiv
\item Aber 18ms Unterschied bei cw+3
	\begin{itemize}
	\item alle Wörter an dieser Stelle waren Adjektive\\
	 $\Rightarrow$ Verlangsamte Lesezeit auf dem zweiten flektierten Adjektiv
	\item ausgelöst durch Übertragungseffekt vom kritischen Adjektiv?
	\end{itemize}
\end{itemize}}



\frame{\frametitle{Zusammenfassung Experiment 3}
\begin{itemize}
\item Effekt durch kritisches Adjektivsuffix, das nicht im Genus mit dem vorgeahnten Nomen übereinstimmt, konnte hier nicht beobachtet werden
\item Dieser Effekt kann allerdings etwas später beim letzten flektierten Adjektiv beobachtet werden
\item großer Effekt bei nicht erwarteten Nomen zu erkennen\\
\end{itemize}
$\Rightarrow$ Leser benutzen ebenfalls Informationen aus dem Kontext um bestimmte Wörter vorauszuahnen 
}

\frame{\frametitle{Zusammenfassung Experiment 3}
\begin{itemize}
\item Versuchspersonen haben keine Manipulation der Daten bemerkt\\
$\Rightarrow$ Vorahnungen, sowie das widerrufen dieser nach Auftreten unpassender Adjektive geschieht unbewusst und routinemäßig
\end{itemize}
}



\subsection{Zusammenfassung der Experimente}
\frame{\frametitle{Zusammenfassung der Experimente}
\begin{itemize}
\item  Experiment 1 und Experiment 3:\\
Mit Kontext Effekte erkennbar, dass Leser und Hörer diesen nutzen um Wörter im Satz vorauszuahnen
\item Experiment 2:\\
Ohne weiteren Kontext keine Effekte erkennbar, dass Hörer diesen nutzen um Wörter im Satz vorauszuahnen
\end{itemize} 
}

\section{Fazit}
\frame{\frametitle{Fazit}
\begin{itemize}
\item Vorahnung spielt beim Sprachverständnis eine große Rolle
\item Leser und Hörer ahnen bestimmte Wörter im Diskurs voraus
\item Leser und Hörer reagieren auf Adjektivsuffixe und Nomen, die nicht mir der Vorahnung übereinstimmen\\
$\Rightarrow$ Sprachverständnissystem funktioniert quasi wie ein Parser
	
\end{itemize}
}

\frame{\frametitle{Vielen Dank}
\centering
\Large{Vielen Dank für Ihre Aufmerksamkeit!}
}

\end{document}

